\documentclass[11pt, a4paper]{book}
\usepackage{svn-multi}
\svnid{$Id$}
\usepackage{prelim2e}
\renewcommand{\PrelimWords}{Draft Copy \svnkw{Id}}
%%\newcommand*{\mysvnrev}{\svnrev}
\usepackage[hyperindex=true,
			bookmarks=true,
            pdftitle={}, pdfauthor={Xi Yang},
            colorlinks=false,
            pdfborder=0,
            pagebackref=false,
            citecolor=blue,
            plainpages=false,
            pdfpagelabels,
            pagebackref=true,
            hyperfootnotes=false]{hyperref}
\usepackage[all]{hypcap}
\usepackage[palatino]{anuthesis}
\usepackage{afterpage}
\usepackage{graphicx}
\usepackage{thesis}
\usepackage[square]{natbib}
\usepackage[normalem]{ulem}
\usepackage[table]{xcolor}
\usepackage{makeidx}
\usepackage{cleveref}
\usepackage[centerlast]{caption2}
\usepackage{float}
\urlstyle{sf}
\renewcommand{\sfdefault}{uop}
\usepackage[T1]{fontenc}
\usepackage[scaled]{beramono}

\usepackage{multirow}


\renewcommand*{\backref}[1]{}
\renewcommand*{\backrefalt}[4]{
  \ifcase #1 %
    %
  \or
    (cited on page #2)%
  \else
    (cited on pages #2)%
  \fi
}


\chapter{Dynamical Systems and Generalised Modelling}
\label{cha:methodology}

\section{Dynamical Systems Theory}
\label{sec:dynamicalIntro}
The mathematics of differential equations provides a powerful set of tools for modelling real world phenomena. In particular, we are interested in using the theory of dynamical systems which is the language of how deterministic systems evolve through time. A one-dimensional dynamical system is described by a first-order differential equation of the form,

\[\frac{dx}{dt} = f(x) \]

The archetypal example is the logistic equation which was constructed by Verhulst (cite) to describe population growth,

\[\frac{dN}{dt} = rN\left(1-\frac{b}{r}N\right) \]



 For example, a two dimensional dynamical system is defined by,

\begin{eqnarray}
\frac{dx}{dt} &= f(x,y) \\
\frac{dy}{dt} &= g(x,y)
\end{eqnarray}




\section{Bifurcation Theory}
\label{sec:bifurcationTheory}

Table~\ref{tab:machines} shows how to include tables and Figure~\ref{fig:helloworld} shows how to include codes.
\begin{table*}
  \centering
  \input table/machines.tex
  \caption{Processors used in our evaluation.}
  \label{tab:machines}
\end{table*}



\begin{figure}
  \centering
  \subfigure[\label{fig:c:hello}]{
  \begin{minipage}[b]{\columnwidth}
    \lstinputlisting[linewidth=\columnwidth,breaklines=true]{code/hello.c}\vspace*{-2ex}
  \end{minipage}}
  \subfigure[\label{fig:java:hello}]{
  \begin{minipage}[b]{\columnwidth}
    \lstinputlisting[linewidth=\columnwidth,breaklines=true]{code/hello.java}\vspace*{-2ex}
  \end{minipage}}
  \caption{Hello world in Java and C.}
  \label{fig:helloworld}
\end{figure}



%%% Local Variables: 
%%% mode: latex
%%% TeX-master: "paper"
%%% End: 

\end{document}