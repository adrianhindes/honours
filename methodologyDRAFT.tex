\documentclass[11pt, a4paper]{book}
\usepackage{svn-multi}
\svnid{$Id$}
\usepackage{prelim2e}
\renewcommand{\PrelimWords}{Draft Copy \svnkw{Id}}
%%\newcommand*{\mysvnrev}{\svnrev}
\usepackage[hyperindex=true,
			bookmarks=true,
            pdftitle={}, pdfauthor={Xi Yang},
            colorlinks=false,
            pdfborder=0,
            pagebackref=false,
            citecolor=blue,
            plainpages=false,
            pdfpagelabels,
            pagebackref=true,
            hyperfootnotes=false]{hyperref}
\usepackage[all]{hypcap}
\usepackage[palatino]{anuthesis}
\usepackage{afterpage}
\usepackage{graphicx}
\usepackage{thesis}
\usepackage[square]{natbib}
\usepackage[normalem]{ulem}
\usepackage[table]{xcolor}
\usepackage{makeidx}
\usepackage{cleveref}
\usepackage[centerlast]{caption2}
\usepackage{float}
\urlstyle{sf}
\renewcommand{\sfdefault}{uop}
\usepackage[T1]{fontenc}
\usepackage[scaled]{beramono}

\usepackage{multirow}


\renewcommand*{\backref}[1]{}
\renewcommand*{\backrefalt}[4]{
  \ifcase #1 %
    %
  \or
    (cited on page #2)%
  \else
    (cited on pages #2)%
  \fi
}

\begin{document}
\chapter{Dynamical Systems and Generalised Modelling}
\label{cha:methodology}

\section{Dynamical Systems Theory}
\label{sec:dynamicalIntro}
The mathematics of differential equations provides a powerful set of tools for modelling real world phenomena. In particular, we are interested in using the theory of dynamical systems which is the language of how deterministic systems evolve through time. A one-dimensional dynamical system is described by a first-order differential equation of the form,

\[\frac{dx}{dt} = f(x) \]

The archetypal example is the logistic equation which was constructed by Verhulst (cite) to describe population growth,

\[\frac{dN}{dt} = rN\left(1-\frac{b}{r}N\right) \]

$N$ is the population, $r$ is the rate of maximum population growth and $K$ the carrying capacity. Although the logistic growth model has a closed form solution - the logistic equation - here we are not interested in solving differential equations. Often it is not possible to find more than a numerical approximation for a differential equation, and the task gets significantly more difficult in higher dimensions with coupled systems. A two-dimensional dynamical system is defined by,

\begin{eqnarray}
\frac{dx}{dt} &= f(x,y) \\
\frac{dy}{dt} &= g(x,y)
\end{eqnarray}

A famous example is the Lotka-Volterra predator-prey model,

\begin{eqnarray}
\frac{dx}{dt} &= ax-dxy \\
\frac{dy}{dt} &= -cy +dbxy
\end{eqnarray}

Where $x$ is the population of prey, $y$ population of predators, $a$ is the birth 


\section{Bifurcation Theory}
\label{sec:bifurcationTheory}



%%% Local Variables: 
%%% mode: latex
%%% TeX-master: "paper"
%%% End: 

\end{document}