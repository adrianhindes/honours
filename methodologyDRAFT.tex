\documentclass[11pt, a4paper]{book}
\usepackage{svn-multi}
\svnid{$Id$}
\usepackage{prelim2e}
\renewcommand{\PrelimWords}{Draft Copy \svnkw{Id}}
%%\newcommand*{\mysvnrev}{\svnrev}
\usepackage[hyperindex=true,
			bookmarks=true,
            pdftitle={}, pdfauthor={Xi Yang},
            colorlinks=false,
            pdfborder=0,
            pagebackref=false,
            citecolor=blue,
            plainpages=false,
            pdfpagelabels,
            pagebackref=true,
            hyperfootnotes=false]{hyperref}
\usepackage[all]{hypcap}
\usepackage[palatino]{anuthesis}
\usepackage{afterpage}
\usepackage{graphicx}
\usepackage{thesis}
\usepackage[square]{natbib}
\usepackage[normalem]{ulem}
\usepackage[table]{xcolor}
\usepackage{makeidx}
\usepackage{cleveref}
\usepackage[centerlast]{caption2}
\usepackage{float}
\urlstyle{sf}
\renewcommand{\sfdefault}{uop}
\usepackage[T1]{fontenc}
\usepackage[scaled]{beramono}

\usepackage{multirow}


\renewcommand*{\backref}[1]{}
\renewcommand*{\backrefalt}[4]{
  \ifcase #1 %
    %
  \or
    (cited on page #2)%
  \else
    (cited on pages #2)%
  \fi
}

\begin{document}
\chapter{Dynamical Systems and Generalised Modelling}
\label{cha:methodology}

\section{Dynamical Systems Theory}
\label{sec:dynamicalIntro}
The mathematics of differential equations provides a powerful set of tools for modelling real world phenomena. In particular, we are interested in using the theory of dynamical systems which is the language of how deterministic systems evolve through time. A one-dimensional dynamical system is described by a first-order differential equation of the form,

\[\frac{dx}{dt} = f(x) \]

The archetypal example is the logistic equation which was constructed by Verhulst (cite) to describe population growth,

\[\frac{dN}{dt} = rN\left(1-\frac{b}{r}N\right) \]

$N$ is the population, $r$ is the rate of maximum population growth and $K$ the carrying capacity. Although the logistic growth model has a closed form solution - the logistic equation - here we are not interested in solving differential equations. Often it is not possible to find more than a numerical approximation for a differential equation, and the task gets significantly more difficult in higher dimensions with coupled systems with two or more equations. Nevertheless, interesting and important mathematical properties of dynamical systems can be ascertained without a closed-form or numerical solution. To explore the relevant basics, we will briefly discuss a two-dimensional predator-prey model known as the Lotka-Volterra equations.


\begin{eqnarray}
\frac{dx}{dt} &= ax-dxy \\
\frac{dy}{dt} &= -cy +dbxy
\end{eqnarray}

Where $x$ is the population of prey, $y$ population of predators. The parameters $a,b,c,d$ describe biological properties of both species, and their dynamics. In particular, $a$ and $d$ are the birth and predation rates for prey, $c$ is the natural mortality rate of predators and $b$ is the growth factor from predation. The most obvious thing to check are fixed points of the system, given by setting the derivatives to zero,

\begin{eqnarray}
x(a-dy) &= 0\\
 -y(c -dby) &= 0
\end{eqnarray}

\section{Bifurcation Theory}
\label{sec:bifurcationTheory}



%%% Local Variables: 
%%% mode: latex
%%% TeX-master: "paper"
%%% End: 

\end{document}